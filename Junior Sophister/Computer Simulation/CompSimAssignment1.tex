\documentclass{article}
\usepackage{graphicx}
\usepackage{mathtools}
\usepackage{xfrac}
\usepackage{amsmath, amssymb}
\usepackage{listings}
\usepackage{float}
\usepackage{wrapfig}
\usepackage{tikz}
\usepackage{fullpage}
\usepackage{hyperref}
\usepackage{mathalpha}
\usepackage{tikz}
\usepackage{cite}
\usepackage{amsthm}

\newtheorem{theorem}{Proposition}[section]
\newtheorem{corollary}{Corollary}[theorem]
\newtheorem{lemma}[theorem]{Lemma}

\theoremstyle{definition}
\newtheorem{definition}{Definition}[section]

\theoremstyle{remark}
\newtheorem*{remark}{Remark}
\newtheorem*{example}{Example}
\newtheorem*{notation}{Notation}

\title{Computer Simulation Assignment 1}
\author{David Lawton\\ Student No.: 22337087}
\date{28th Sep. 2024.}

\begin{document}

\maketitle

\tableofcontents

\newpage

\section{Heron's Root Finding Method}
\section{Numerical Accuracy: Landau and Paez, section 1.3 questions}

\subsection{1.3 Question 1}

\subsubsection{Part A}

\subsubsection{Part B}

\subsection{1.3 Question 2}

\subsubsection{Part A}

\subsubsection{Part B}

\subsection{1.4 Question 1}

\subsubsection{Part A}

\subsubsection{Part B}




\end{document}