\documentclass{article}
\usepackage{graphicx}
\usepackage{mathtools}
\usepackage{xfrac}
\usepackage{amsmath, amssymb}
\usepackage{listings}
\usepackage{float}
\usepackage{wrapfig}
\usepackage{tikz}
\usepackage{fullpage}
\usepackage{hyperref}
\usepackage{mathalpha}
\usepackage{tikz}
\usepackage{cite}
\usepackage{amsthm}
\usepackage{braket}
\usepackage[utf8]{inputenc}
\usepackage[T1]{fontenc}


\newtheorem{theorem}{Proposition}[section]
\newtheorem{corollary}{Corollary}[theorem]
\newtheorem{lemma}[theorem]{Lemma}

\theoremstyle{definition}
\newtheorem{definition}{Definition}[section]

\theoremstyle{remark}
\newtheorem*{remark}{Remark}
\newtheorem*{example}{Example}
\newtheorem*{notation}{Notation}

\title{UTRIP Personal Statement}
\author{David Lawton}
\date{7th Jan. 2024}

\begin{document}
\maketitle
My name is David Lawton, I am a third year undergraduate at Trinity College, Dublin, Ireland studying toward a BA in Theoretical Physics. I am writing this statement to summarise my reasons for applying and my qualifications for the projects to which I am applying.\\
\indent In regards to my reasons for applying to the particular research projects I selected, namely those under Prof. Toshikazu Shigeyama, and Associate Prof. Haozhao Liang, I would say that my motivations are numerous.\\
\indent I will begin by discussing my first choice, the project under Prof. Toshikazu Shigeyama. I found the goal of the group particularly interesting, to clarify the origins of the elements synthesized throughout the history of the universe. The possibility of being involved in such significant work is extremely motivating, and I felt I had to apply. I am personally more interested in theoretical physics than experimental, and this pushed me to apply for this project ahead of many others. I thought it would be prudent to read some of Prof. Shigeyama's work, and so took a look at his paper \textbf{`Nuclear Burning in Accretion Flow of Helium-rich matter onto Compact Objects'} with Akira Dohi, since the title caught my eye as seeming interesting. I found the comparison of the results for the He-rich matter with those of CO matter studied in Nagarajan \& Shigeyama (2022) to be a highlight, and the suggestion of work focused on high-specific-enthalpy models in which the nuclear burning being accounted is not limited to the supersonic region sounded like not only a fascinating theoretical endeavour, but a fascinating computational one as well.\\
\indent My second choice, the project under Assoc. Prof. Haozhao Liang regarding `\textbf{quantum many-body theories for properties of atomic nuclei}' is similarly compelling. I would jump at the idea of applying the quantum mechanics I have been learning over the past few years, and the quantum mechanics I am yet to learn, to either a computational or theoretical project, as I have found it to be one of my favourite physics subjects so far.\\
\indent What I hope to gain from the overall experience of the UTRIP program is not only the research experience in such riveting topics, and the ability to use the material I have studied during my degree, but the cultural experience, and the personal development I might gain by encountering a culture so contrasted to my own.\\
\indent On the subject of the project under Prof. Shigeyama, I have studied, or will in the coming semester study, all the topics listed to some depth. I am particularly interested in statistical physics, which I have been doing a large quantity of problems in on top of the work done in college, and fluid dynamics, on which a few of my classmates and I were given a short series of private lectures by one of our lecturers. In regards to knowledge of PDEs, while I have often encountered them, and methods of solving them, over the last two years, the module I am taking this coming semester, `Linear PDEs', should cover any knowledge of PDEs required. I also think my experience with numerical methods, and computational physics would be useful for any simulations being done.\\
\indent With regards to the second project, by the summer I will have studied significantly more QM than I have to date. Currently, I have covered topics including, but not limited to, Hilbert spaces \& creation and annihalition operators, tensor products of hilbert spaces, algebras and representations, postulates of QM, QM dynamics, symmetries, harmonic oscillators, delta-function well, motion in EM field and angular momentum \& central fields. As well as this I have experience with applications of numerical methods in physics simulation in Python and in C++. Since the project is focused on atomic physics, I will also mention the module, Atomic Physics \& Statistical Thermodynamics, that I will be taking next semester, which I'm sure will give me more knowledge on the subject than previous general physics modules have.\\
\indent Over the next semester, along with the aforementioned Linear PDEs module, I will be taking Statistical Physics II (quantum statistical physics, spin systems, mean field analysis), Quantum Mechanics II (Time-dependent/independent perturbation theory, path-integral formulation, quantum measurement, possibly scattering), Atomic Physics \& Statistical Thermodynamics and Classical Electrodynamics.\\
\indent While I haven't put an enormous amount of thought into which graduate programs I might apply for over the next year, I will certainly be considering applying to the Graduate School of Sciences at the University of Tokyo. In part because of the prestige of the university in relation to physics, but also because of the active research into topics I am interested in pursuing, such as condensed matter theory and astroparticle physics, the relative attractiveness of a big city like Tokyo, compared to my hometown of Cork, Ireland. I would be remiss not to say that I have a certain desire to leave my home country once I have completed my BA, and gain some life experience.\\
\indent Thank you for your considerations.
\end{document}

([A] 4.0 grade points x 50 credits + [B] 3.0 grade points x 50 + [C] 2.0 grade points x 20 credits)/(120) = GPA 3.503 out of 4.00. All grades included, on conversion, First Class - A, Second Class - B, Third Class - C.
