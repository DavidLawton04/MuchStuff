
\documentclass{letter}
\usepackage[utf8]{inputenc}
\usepackage[T1]{fontenc}
\usepackage{geometry}
\usepackage{fancyhdr}
\usepackage[colorlinks=true, urlcolor=blue]{hyperref}

% Set the page margins
\geometry{top=1in, bottom=1in, left=1.5in, right=1.5in}

% Set the header and footer
\pagestyle{fancy}
\fancyhf{} % Clear all header and footer fields
\renewcommand{\headrulewidth}{0pt} % Remove the header rule line
\rhead{David Lawton}
\cfoot{\thepage}
\signature{David Lawton.}

\begin{document}

\begin{letter}{RPTU Kaiserslautern-Landau\\
    Department of Physics\\
    Pelster group\\}

\opening{To whom it may concern,}
I am writing to you to apply for the projects `\textit{Quantum Phase Transition of the Transverse-Field Ising Chain}', and `\textit{Bose-Einstein Condensate: Quantum System Out of Equilibrium}' at IFSC-USP. My name is David Lawton, and I am a Junior Sophister student at Trinity College Dublin studying Theoretical Physics.\\

I am applying for the project `\textit{Quantum Phase Transition of the Transverse-Field Ising Chain}' as, on reading the description, and doing some brief reading on the topics mentioned, I found that it correlated well with the content of both the quantum mechanics, and statistical mechanics courses which I have recently completed. While I have not seen previously Jordan-Wigner transformations specifically, the operators and notations used are familiar. As well as this, phase transitions were covered from thermodynamics and statistical mechanics viewpoints the semester past.\\

I am applying for the project `\textit{Bose-Einstein Condensate: Quantum System Out of Equilibrium}' due to the interest I have in studying condensed matter theory in the future. The potential applications of further study of superconductivity as a whole, is in my opinion, both fascinating and essential. In addition, while I might be a theoretical physics undergraduate, I do have some laboratory experience as part of my physics modules, and would be interested in learning more about the experimental side of physics.\\

I believe that the modules which I have taken during my studies have provided me with a solid basis of knowledge in theoretical physics, as well as some mathematics. Over the past year, with the School of Maths, I have taken modules including, but not limited to, advanced classical mechanics, quantum mechanics, statistical mechanics and classical field theory. I have also taken modules in the our School of Physics, such as computer simulation, physics for theoretical physics (general physics modules) and condensed matter physics. As well as this, I will be taking a module in electrodynamics, while further studying quantum mechanics, statistical mechanics and condensed matter, during the next semester. Further details pertaining to these modules can be found online \href{https://www.maths.tcd.ie/undergraduate/modules/jstp.php}{here}. The material of both the quantum mechanics and statistical mechanics modules, as well as condensed matter physics, being the most relevant to the projects.\\

I would also like to note, as there is no specific place to do so on the application form, that I have experience in Python, especially as applied to physics, Mathematica, and Jupyter, as well as proficiency in \LaTeX.\\

\closing{Yours sincerely,}

\end{letter}
\end{document}