\documentclass{article}
\usepackage{graphicx}
\usepackage{mathtools}
\usepackage{xfrac}
\usepackage{amsmath, amssymb}
\usepackage{listings}
\usepackage{float}
\usepackage{wrapfig}
\usepackage{tikz}
\usepackage{fullpage}
\usepackage{hyperref}
\usepackage{mathalpha}
\usepackage{tikz}
\usepackage{cite}
\usepackage{amsthm}
\usepackage{braket}
\usepackage[utf8]{inputenc}
\usepackage[T1]{fontenc}


\newtheorem{theorem}{Proposition}[section]
\newtheorem{corollary}{Corollary}[theorem]
\newtheorem{lemma}[theorem]{Lemma}

\theoremstyle{definition}
\newtheorem{definition}{Definition}[section]

\theoremstyle{remark}
\newtheorem*{remark}{Remark}
\newtheorem*{example}{Example}
\newtheorem*{notation}{Notation}

\title{PLANCKS Lecture I: Hydrodynamics/Fluid Mechanics\\ Based on lectures given by Dr. Chaolun Wu}
\author{David Lawton}
\date{19.12.2024}

\begin{document}

\maketitle
\newpage
\tableofcontents
\newpage
\section{Introduction}
Fluid Mechanics, or Hydrodynamics, can be defined as a \textbf{low-energy effective description of a many-body dynamical system}. This system could be governed by either classical or quantum mechanics.\\
From first principles, we have conservation laws for Noether charges. We have two equations of motion:
\begin{equation}
    \partial_\mu J^\mu = 0 \quad \partial_\mu T^{\mu\nu} = 0 \quad(Rel/NR)
\end{equation}
where $J^\nu$ is the charge current, and $T^{\mu\nu}$ is the energy-stress tensor.\\
\indent The system itself has 4 variables: $u^\mu(x)$ or $\vec{v}(\vec{x},t), \epsilon(\vec{x}, t), P(\vec{x},t), \rho(\vec{x}, t)$. These correspond to four-velocity or velocity, energy density, pressure and mass density respectively.\\
\indent We aim to find `\textbf{constituent relations}', which are $J^\nu,T^{\mu\nu}$ as functions of the system variables $u^\mu, P, \dots$. The way in which we find these is why we call this an \textbf{effective} description. Since any terms allowed by the symmetries of the system will exist, we know that certain terms are contained in each of the relations, e.g.
\begin{equation}
J^\nu \sim u^\mu,\quad T^{\mu\nu} \sim u^\mu u^\nu, g^{\mu\nu}
\end{equation}
index notation implies the relations are manifestly covariant.\\
\indent Since our description is \textbf{low-energy}, we organise our constituent relations using a derivative expansion as a Taylor series, where we can charecterise our spacial/temporal variation/fluctuation by
\begin{equation}
    \vec{p}\sim\vec{\partial},\quad E \sim \partial_t
\end{equation}
and our charecteristic scales of length, time are the mean free path $l_{\mathrm{mfp}}$ and relaxation time $T_{\mathrm{rel}}$ respectively. In a \textbf{low-energy} description, we can say $l_{\mathrm{mfp}}|\vec{\partial}|, T_{\mathrm{rel}} \ll 1$





\end{document}