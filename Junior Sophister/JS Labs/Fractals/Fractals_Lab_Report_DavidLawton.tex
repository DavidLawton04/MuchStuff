\documentclass{article}
\usepackage{graphicx}
\usepackage{mathtools}
\usepackage{xfrac}
\usepackage{amsmath, amssymb}
\usepackage{listings}
\usepackage{float}
\usepackage{wrapfig}
\usepackage{tikz}
\usepackage{fullpage}
\usepackage{hyperref}
\usepackage{mathalpha}
\usepackage{tikz}
\usepackage{cite}
\usepackage{amsthm}


\setcounter{section}{-1}

\newtheorem{theorem}{Proposition}[section]
\newtheorem{corollary}{Corollary}[theorem]
\newtheorem{lemma}[theorem]{Lemma}

\theoremstyle{definition}
\newtheorem{definition}{Definition}[section]

\theoremstyle{remark}
\newtheorem*{remark}{Remark}
\newtheorem*{example}{Example}
\newtheorem*{notation}{Notation}

\title{Fractals}
\author{David Lawton\\
        Lab Partner: Sami Lopez-Steffenson\\
        22337087}
\date{14th Oct. 2024.}

\begin{document}

\maketitle

\tableofcontents
\addcontentsline{toc}{section}{\numberline{}Abstract}
\begin{abstract}
        
\end{abstract}

\section{Keywords \& Preliminaries}
\begin{definition}
        An \textbf{`ideal' fractal} is a scale independent geometric object. Scale independent meaning that the scale on which the object is viewed does not affect the appearance.\cite{LabHandbook}
\end{definition}
\begin{definition}
        A \textbf{`real' fractal} is a physical object which resembles a fractal one over certain scales. However, the object size sets an upper limit on the scale at which the fractal properties are observed, and of course, the resolution sets, a less defined, lower bound.\cite{LabHandbook}
\end{definition}
\begin{definition}
        The \textbf{`fractal dimension'} of an object is a measure of the 
\section{Background \& Theory}
This labs concerns the analysis of fractal growth under varying conditions. The fractals in this experiment were grown using zinc sulphate solutions. The fractals were grown in a thin plastic dish, with a plastic top. There is a conducting ring on the edge of the dish, and a thin stick of graphite in the centre, with a voltage applied across. Measurements were taken, with both voltage across the solution and molarity of the solution being varied.







\section{Procedure}



\section{Results}



\section{Discussion}



\addcontentsline{toc}{section}{\numberline{}Appendix}
\section*{Appendix}



\bibliographystyle{plain}
\bibliography{references}

\end{document}