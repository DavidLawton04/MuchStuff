\documentclass{article}
\usepackage{graphicx}
\usepackage{mathtools}
\usepackage{xfrac}
\usepackage{amsmath, amssymb}
\usepackage{listings}
\usepackage{float}
\usepackage{wrapfig}
\usepackage{tikz}
\usepackage{fullpage}
\usepackage{hyperref}
\usepackage{mathalpha}
\usepackage{tikz}
\usepackage{cite}
\usepackage{amsthm}
\usepackage{natbib}
\usepackage{multirow}
\usepackage[T1]{fontenc}
\setcounter{section}{-1}

\newtheorem{theorem}{Proposition}[section]
\newtheorem{corollary}{Corollary}[theorem]
\newtheorem{lemma}[theorem]{Lemma}

\theoremstyle{definition}
\newtheorem{definition}{Definition}[section]

\theoremstyle{remark}
\newtheorem*{remark}{Remark}
\newtheorem*{example}{Example}
\newtheorem*{notation}{Notation}

\title{Numerical Solution of the Time-Independent 1D Schr\"{o}dinger Equation}
\author{David Lawton\\
        22337087}
\date{4th Nov. 2024.}

\begin{document}

\maketitle

\tableofcontents
\addcontentsline{toc}{section}{\numberline{}Abstract}
\begin{abstract}

\end{abstract}

\section{Keywords \& Preliminaries}
\begin{list}{Keywords}{0.5cm}
    \item Schr\"{o}dinger Equation
    \item Eigenvalues
\end{list}


\section{Background \& Theory}
In this computational laboratory, we shall be solving the time-independent Schr\"{o}dinger equation for a particle in a one-dimensional potential well. The time-independent Schr\"{o}dinger equation is 
\begin{equation}
    -\frac{\hbar^2}{2m}\frac{\mathrm{d}^2\psi(x)}{\mathrm{d}x^2} + V(x)\psi(x) = E\psi(x)
\end{equation}
where $\psi(x)$ is the wavefunction of the particle, $V(x)$ the potential the particle is under, $m$ the mass of the particle, $E$ the energy of the particle and $\hbar$ the reduced Planck constant. In this lab, we shall remove the dimensions from this equation, to avoid computation of excessively small numbers. Thus we get the non-dimensional Schr\"{o}dinger equation.This is given by
\begin{equation}
    \frac{\mathrm{d}^2\psi(\tilde{x})}{\mathrm{d}\tilde{x}^2} + \gamma^2(\varepsilon-\nu(\tilde{x}))\psi(\tilde{x}) = 0
\end{equation}
where our non dimensional constants, variables and functions are $\tilde{x} = x/L$, $\varepsilon = E/V_0$, $\nu(\tilde{x}) = V(\tilde{x})/V_0$ and 
\begin{equation}
    \gamma^2 = \frac{2mL^2V_0}{\hbar^2}
\end{equation}
\subsection{Discretization of Wavefunction and its Derivatives}
To find a numerical solution for this equation, we first discretize our continuous coordinate $\tilde{x}$ and wavefunction $\psi(\tilde{x})$ into $N$ points. Thus each point is defined by $\tilde{x}_n = \frac{n}{N}$, $n = 0, 1, ..., N$. To discretize our wavefunction, we Taylor expand $\psi(\tilde{x}\pm l), l=1/(N-1)$ about $\tilde{x}$ up to fourth order, and adding the two expansions giving us
\begin{equation}
    \psi(\tilde{x}+l) + \psi(\tilde{x}-l) = 2\psi(\tilde{x}) + l^2\psi^{\prime\prime}()\tilde{x}) + \frac{l^4\psi^{(4)}(\tilde{x})}{12} + O(l^6)
\end{equation}
\section{Methodology}

\section{Results}


\subsection{Analytic Solution of non-dimensional Schr\"{o}dinger Equation}
Our first task is to solve the non-dimensional Schr\"{o}dinger equation analytically, so that we can later verify our computational results. We must solve for the potential well defined by
\begin{equation}
    \nu(\tilde{x}) = 
        \begin{cases}
            -1 & \text{ if } 0<\tilde{x}<1\\
            \infty & \text{ otherwise}
        \end{cases}
\end{equation}
To do this we first write our non-dimensional Schr\"{o}dinger equation
\begin{equation}
    \frac{\mathrm{d}^2\psi(\tilde{x})}{\mathrm{d}\tilde{x}^2} + \gamma^2(\varepsilon+1)\psi(\tilde{x}) = 0
\end{equation}
since $\varepsilon$ has no $\tilde{x}$ dependence, the solution is simply 
\begin{equation}
    \psi(\tilde{x}) = 
        \begin{cases}
            Ae^{ik\tilde{x}} + Be^{-ik\tilde{x}} & \text{ if }x\in(0,1)\\
            0 & \text{ otherwise}
        \end{cases}
\end{equation}
where $k = \gamma\sqrt{\varepsilon+1}$. We let $\psi_M(\tilde{x}) = \psi(\tilde{x})$ for $x\in (0,1)$. Then, since $\psi$ must be smooth, we can say $\psi(0)=\psi(1)=0$. This gives us $A+B=0$, so $A=-B$. Thus we have
\begin{equation}
    \psi_M(\tilde{x}) = A(e^{ik\tilde{x}}-e^{-ik\tilde{x}}) = 2iA\sin(k\tilde{x}) = C\sin(k\tilde{x}).
\end{equation}
We then use our second boundary condition, at $\tilde{x}=1$, giving us $\sin(kx) = 0$. This gives us $k_n = n\pi$ for $n\in\mathbb{N}$. To find the solutions we then normalise the wavefunction (integrate $|\psi|^2$ and equate to 1), giving us $C=\sqrt{2}$. Thus we have our analytic solutions
\begin{equation}
    \psi_n(\tilde{x}) = \sqrt{2}\sin(n\pi\tilde{x})
\end{equation}
from our equivalent $k_n$ definition we solve for our energy values $\varepsilon_n$,
\begin{equation}
    \varepsilon_n = \frac{n^2\pi^2}{\gamma^2}-1
\end{equation}
\end{document}